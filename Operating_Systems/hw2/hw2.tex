% Created 2023-02-02 Thu 08:30
% Intended LaTeX compiler: lualatex
\documentclass[11pt]{article}
\usepackage{graphicx}
\usepackage{longtable}
\usepackage{wrapfig}
\usepackage{rotating}
\usepackage[normalem]{ulem}
\usepackage{amsmath}
\usepackage{amssymb}
\usepackage{capt-of}
\usepackage{hyperref}
\usepackage{minted}
\usepackage{physics}
\author{David Lewis}
\date{\today}
\title{}
\hypersetup{
 pdfauthor={David Lewis},
 pdftitle={},
 pdfkeywords={},
 pdfsubject={},
 pdfcreator={Emacs 28.2 (Org mode 9.6)}, 
 pdflang={English}}
\begin{document}

\tableofcontents

\section{1.4}
\label{sec:orgbec6297}
False: a tainted kernel can work perfectly fine. The ``taint'' comes from a list
of different reasons that may cause problems down the line. This is useful for debugging.
\section{1.5}
\label{sec:org8f94e1a}
The current tainted state is recorded at \texttt{/proc/sys/kernel/tained}. When the kernel
is not tained the taint state is 0. To view the current state, one can use the
following command \texttt{cat /proc/sys/kernel/tainted}. This results in a decimal number
corresponding to the various taints in the kernel. Converting this decimal
number to binary and examining the bits, one can determine the various taints. I
wrote the following script to do just that.

\begin{minted}[fontsize=\scriptsize]{bash}
#!/usr/bin/env bash

# this function is a little hard to explain, but essentially
# the >> operator appends the tainted state to standard out
# but the number gets bit shifted before hand, resulting in the
# output of the binary digit specified by $1
function tainted_state(){
    echo $(($(cat /proc/sys/kernel/tainted)>>($1)&1))
}

# This is what each bit means if it has the value of 1
# I used the table from: https://docs.kernel.org/admin-guide/tainted-kernels.html#more-detailed-explanation-for-tainting
# as reference
problems=( "G/P: proprietary module was loaded"
           "F: module was force loaded"
           "S: kernel running on an out of specification system"
           "R: module was force unloaded"
           "M: processor reported a Machine Check Exception (MCE)"
           "B: bad page referenced or some unexpected page flags"
           "U: taint requested by userspace application"
           "D: kernel died recently, i.e. there was an OOPS or BUG"
           "A: ACPI table overridden by user"
           "W: kernel issued warning"
           "C: staging driver was loaded"
           "I: workaround for bug in platform firmware applied"
           "O: externally-built (out-of-tree) module was loaded"
           "E: unsigned module was loaded"
           "L: soft lockup occurred"
           "K: kernel has been live patched"
           "X: Auxiliary taint, defined for and used by distros"
           "T: kernel was built with the struct randomization plugin"
           "N: an in-kernel test has been run"
         )

# there are 19 bits 0-18. If the bit is 1, the corresponding error is outputted.
for i in {0..18}
do
    if [[ $(tainted_state $i) == "1" ]]; then
        echo ${problems[$i]}
    fi
done
\end{minted}
*
\end{document}
